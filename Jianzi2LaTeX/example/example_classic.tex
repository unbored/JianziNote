\documentclass{zhvt-classic}

\usepackage{jianzinote}

\zhvtset{
  % 正文字体,思源宋体中等
  main font           = Source Han Serif K,
  % 夹注字体,思源宋体中等
  jiazhu font         = Source Han Serif K,
  % 字体尺寸
  font size           = 20pt,
  % 行距
  baseline skip       = 38pt,
  % 书宽,A5
  book width          = 148mm,
  % 书高,A5
  book height         = 210mm,
  % 每页行数
  page lines          = 9,
  % 每行字数
  line chars          = 20,
  %
  adjust ratio        = 0.62,
  % 半高夹注
  jiazhu half size    = true,
}

\title{酒狂}[酒狂]

\begin{document}

\chapter*{酒狂}[神奇秘譜\\宮調]

\begin{preface}
臞仙曰是曲者阮籍所作也籍嘆道之不行與時不合
故忘世慮於形骸之外托興於酗酒以樂於終身之志
其趣也若是豈真嗜於酒耶有道存焉玅玅於其中
故不落俗子道達者得之
\end{preface}

\section{其一}
\encircle{狂}
\jzr{散打三,大九挑六}
\jiazhu{\jzr{虚,抓,起}}
\jzr{散打三,大七九挑六}
\jiazhu{\jzr{虚,抓,起}}
\jzr{散打三,大七抹六,就撮六三}%
\jzr{散打三,就挑六}
\jiazhu{\jzr{虚,抓,起}}
\jzr{散打三,大六四挑六}
\jiazhu{\jzr{虚,抓,起}}
\jzr{散打三,大五六抹六,就撮六三}
\encircle{歌}
\jiazhu{\jzr{从}狂\jzr{再,作}}
\encircle{飲}
\jzr{散打三,大五六抹六}
\jiazhu{\jzr{虚,抓,起}}
\jzr{散打三,大六四挑六}
\jiazhu{\jzr{虚,抓,起}}
\jzr{散打三,大七抹六,就撮六三}%
\jzr{散打三,大七挑六}
\jiazhu{\jzr{虚,抓,起}}
\jzr{散打三,大七九挑六}
\jiazhu{\jzr{虚,抓,起}}
\jzr{散打三,名十抹六,就撮六三}%
\encircle{仙}
\jzr{(右反复*'名十打摘五'),大九历七六,名十掐起}%
\jzr{散打三,大七抹六,就撮六三}
\jiazhu{\jzr{从,右反复,再,作}}
\encircle{酒}

\section{其二}
\jzr{散打三,大九挑六}
\jiazhu{\jzr{抑,上,七,九}}
\jzr{散打三,大七九挑六}
\jiazhu{\jzr{抑,上,七}}
\jzr{散打三,大七抹六,就撮六三}%
\jzr{散打三,就挑六}
\jiazhu{\jzr{抑,上,六,四}}
\jzr{散打三,就挑六}
\jiazhu{\jzr{抑,上,五,六}}
\jzr{散打三,就抹六,就撮六三}%
再自\encircle{飲}字彈至\encircle{酒}字

\section{其三}
\jzr{(右反复*'散打五'),大九挑七}
\jiazhu{\jzr{虚,抓,起}}
\jzr{散打四,大九挑六}
\jiazhu{\jzr{虚,抓,起}}
\jzr{散打三,大九抹五,就撮五三}
\jiazhu{\jzr{从,右反复,再,作,抑,上,七,六}}
\jzr{(右反复*'中外绰打一')}
\jiazhu{\jzr{上,十}}
\jzr{散挑三,大九打二,散挑五,中外绰打三}
\jiazhu{\jzr{上,十}}
\jzr{散抹六,撮大九三散六}
\jiazhu{\jzr{从,右反复,再,作}}
自\encircle{仙}至\encircle{酒}再作

\section{其四}
从\encircle{狂}至\encircle{歌}
\encircle{天}
\jzr{(右反复*'中七绰打一')}
\jiazhu{\jzr{上,五,六}}
\jzr{散挑三,大五打二,散挑五,跪七绰打三}
\jiazhu{\jzr{上,五,六}}
\jzr{散抹六,撮大五三散六}
\jiazhu{\jzr{从,右反复,再,作}}
\jzr{(右反复*'跪五六打摘五'),大五历七六,跪五六掐起}%
\jzr{散打三,跪五六抹六,就撮六三}
\jiazhu{\jzr{从,右反复,再,作}}
\encircle{地}%
\jzr{(右反复*'散打五'),大五挑七}
\jiazhu{\jzr{虚,抓,起}}
\jzr{散打四,大五挑六}
\jiazhu{\jzr{虚,抓,起}}
\jzr{散打三,大四八抹五,就撮五三}
\jiazhu{\jzr{从,右反复,再,作,抑,上,四,四}}
从\encircle{天}至\encircle{地}又从\encircle{饮}至\encircle{酒}再作%
\jzr{(右反复*'散挑七'),大九打四,散历七六,大九打三,名十掐起,散历五四,擘六,中十打一,散抹三,撮中十一散三}
\jiazhu{\jzr{从,右反复,再,作}}

\section{仙人吐酒聲}
\jzr{大九按三,散拂一,至三}
\jiazhu{\jzr{长,锁}臨了\jzr{急,锁}三声乘声上七}
\jzr{就按,拂一,至三}
\jiazhu{\jzr{又,长,锁}臨了\jzr{急}三声\jzr{抓,起}}
\jzr{散涓二三,打二,历五四,擘六,中外绰打一}
\jiazhu{\jzr{上,十}}
\jzr{散抹三,就撮三一}終

\end{document}